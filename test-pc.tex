\documentclass[A4paper]{article}
\usepackage[utf8]{inputenc}
\usepackage[left=2cm,right=2cm,top=3cm,bottom=3cm]{geometry}
\usepackage[brazil]{babel}
\usepackage{indentfirst}
\usepackage{url}
\usepackage{listings}

\title{LPOO - Manutenção de Máquinas para Teste}
\author{Raniere Silva \\ \url{ra092767@ime.unicamp.br} \and
        Abel Siqueira \\ \url{abel.s.siqueira@gmail.com}}
\begin{document}
\maketitle

% Este arquivo descreve o ambiente de teste do LPOO.
%
% Ele é reutilizado nos arquivos `test-pc.tex` e `test-run.tex`.
\section{Ambiente}
O LPOO possui disponíveis para a realização de testes computacionais as máquinas
presentes na tabela abaixo.
\begin{table}[h]
  \centering
  \begin{tabular}{|c|c|c|c|}
    \hline
    Hostname & Processador & Memória RAM & Placa de Vídeo \\ \hline
    lpoo-13631 & Intel(R) Core(TM) i7 920 & 8GB & GeForce 9800 GT \\ \hline
  \end{tabular}
\end{table}

As máquinas de teste do LPOO utilizam como sistema operacional o GNU/Linux na forma da
distribuição Arch Linux e seu acesso é exclusivamente via SSH.

Os seguintes softwares encontram-se instalados:
\begin{itemize}
  \item \lstinline+gfortran+
  \item \lstinline+gcc+
  \item \lstinline!g++!
  \item \lstinline+python3+
\end{itemize}

\section{Pré-Requisitos}
Para configurar uma máquina será preciso:
\begin{itemize}
  \item conexão com a internet,
  \item um pen-drive vazio.
\end{itemize}

\section{Pré-Instalação}
Baixe a imagem do Arch Linux em \url{https://www.archlinux.org/download/}.
Depois de baixar a imagem, copie-a para o pen-drive utilizando
\begin{lstlisting}
# dd if=caminho/para/image/Arch/Linux of=/dev/sdX
\end{lstlisting}
onde \lstinline+sdX+ é o disco correspondente ao pen-drive. Caso você não saiba
qual é o disco correspondente ao pen-drive, utilize
\begin{lstlisting}
# fdisk -l
\end{lstlisting}
para descobrir.

Com o pen-drive contendo a imagem do Arch Linux, conecte o pen-drive na máquina
a ser configurada, e ligue-a. Logo que a máquina começar a funcionar você deve
apertar alguma tecla para entrar na BIOS (provavelmente \lstinline+F1+,
\lstinline+F2+ \lstinline+F11+, \lstinline+F12+ ou \lstinline+Del+).

Na BIOS você precisa habilitar o boot pela porta USB (e o dispositivo de rede).
Opcionalmente pode-se configurar a ordem de boot dos dispositivos.

Uma vez configurada a BIOS adequadamente, salve as configurações e saia. A
máquina irá reiniciar e logo que voltar a funcionar você deve apertar alguma
tecla para selecionar o dispositivo a partir do qual será feito o boot
(provavelmente \lstinline+F10+).

Irá aparecer uma tela mostrando algumas opções. Antes de instalar, é recomendado
verificar a integridade do disco de instalação.

\section{Instalação}
A instalação do Arch Linux ocorre toda pelo terminal e iremos mostrar quais
comandos você deve utilizar.

\subsection{Configurando teclado}
Se estiver utilizando um teclado ABNT2 você terá que informá-lo:
\begin{lstlisting}
# loadkeys br-abnt2
\end{lstlisting}
Também é recomendado modificar a fonte:
\begin{lstlisting}
# setfont Lat2-Terminus16
\end{lstlisting}

\subsection{Conectando com a Internet}
Verifique se possue conexão de internet:
\begin{lstlisting}
# ping -c 10 8.8.8.8
\end{lstlisting}

\subsection{Particionamento do disco}
Precisamos particionar o disco, considerando um disco de 500GB utilizaremos a
seguinte configuração:
\begin{itemize}
  \item 50GB para o \lstinline+root+,
  \item 1GB para o \lstinline+boot+,
  \item 16GB para a \lstinline+swap+,
  \item ~400GB para o \lstinline+home+ (é utilizado o resto do disco).
\end{itemize}

Muito provavelmente o disco a ser utilizado é \lstinline+/dev/sda+, se não tiver
certeza ou quiser verificar utilize:
\begin{lstlisting}
# fdisk -l
\end{lstlisting}

Para realizar o particionamento iremos utilizar o \lstinline+gdisk+ (se quiser
algo mais visual utilize o \lstinline+cgdisk+).
\begin{lstlisting}
# gdisk /dev/sda
: o
: n
:
:
: +50G
:
: c
: root
: n
:
:
: +1G
:
: c
: 2
: boot
: n
:
:
: +16G
: 8200
: c
: 3
: swap
: n
:
:
:
:
: c
: 4
: home
: w
\end{lstlisting}
Se precisar de ajuda digite \lstinline+?+ na prompt do \lstinline+gdisk+.

\subsection{Formatação das partições}
Depois de criado as partições, é preciso formatá-las.
\begin{lstlisting}
# mkfs -t ext4 /dev/sda1
# mkfs -t ext4 /dev/sda2
# mkfs -t ext4 /dev/sda4
\end{lstlisting}

Para a swap,
\begin{lstlisting}
# mkswap /dev/sda3
# swapon /dev/sda3
\end{lstlisting}

\subsection{Montagem das partições}
É necessário montar as partições.
\begin{lstlisting}
# mount /dev/sda1 /mnt
# mkdir /mnt/boot
# mount /dev/sda2 /mnt/boot
# mkdir /mnt/home
# mount /dev/sda4 /mnt/home
\end{lstlisting}

\subsection{Instalando o Arch Linux}
Antes de instalar edite o arquivo \lstinline+/etc/pacman.d/mirrorlist+ (você
terá que fazer isso utilizando o \lstinline+vi+ ou o \lstinline+nano+) para
tentar utilizar um servidor mais próximo (mova para o início do arquivo os
servidores que encontram-se no Brasil). As primeiras linhas devem ser parecidas
com
\begin{lstlisting}
Server = ftp://ftp.las.ic.unicamp.br/pub/archlinux/$repo/os/$arch
Server = http://www.las.ic.unicamp.br/pub/archlinux/$repo/os/$arch
Server = http://pet.inf.ufsc.br/mirrors/archlinux/$repo/os/$arch
Server = ftp://archlinux.c3sl.ufpr.br/archlinux/$repo/os/$arch
Server = http://archlinux.c3sl.ufpr.br/$repo/os/$arch
\end{lstlisting}

Para instalar, utilize
\begin{lstlisting}
# pacstrap /mnt base base-devel
\end{lstlisting}

\subsection{Configurando o sistema}
É preciso gerar o arquigo \lstinline+fstab+:
\begin{lstlisting}
# genfstab -p -U /mnt >> /mnt/etc/fstab
\end{lstlisting}

Entre no novo sistema
\begin{lstlisting}
# arch-chroot /mnt
\end{lstlisting}
e configure novamente o teclado. Para salvar a configuração do teclado, adicione
as linhas abaixo ao arquivo \lstinline+/etc/vconsole.conf+:
\begin{lstlisting}
KEYMAP=br-abnt2
FONT=Lat2-Terminus16
\end{lstlisting}
Também configure novamente a conexão de internet.

Configure o hostname:
\begin{lstlisting}
# echo lpoo-XXXXX > /etc/hostname
\end{lstlisting}
onde \lstinline+XXXXX+ é o PI da máquina.

Crie um link simbólico para o fuso horário de Brasília.
\begin{lstlisting}
# ls -s /usr/share/zoneinfo/America/Sao_Paulo /etc/localtime
\end{lstlisting}

Edite o arquivo \lstinline+/etc/locale.gen+ removendo o \lstinline+#+ das linhas
\begin{lstlisting}
#en_US UTF-8 UTF-8
#pt_BR UTF-8 UTF-8
\end{lstlisting}
Depois de editar as linhas, execute
\begin{lstlisting}
# locale-gen
# echo LANG=en_US.UTF-8 > /etc/locale.conf
# export LANG=en_US.UTF-8
\end{lstlisting}

Configure uma senha para o usuário \lstinline+root+:
\begin{lstlisting}
# passwd root
\end{lstlisting}

\subsection{Instalando o bootloader}
Utilizaremos o Syslinux, para instalá-lo:
\begin{lstlisting}
# pacman -S syslinux
\end{lstlisting}

Execute
\begin{lstlisting}
# pacman -S gptfdisk
# syslinux-install_update -i -a -m
\end{lstlisting}

Edite o arquivo
\lstinline+/boot/syslinux/syslinux.cfg+ de forma a ter as seguintes linhas:
\begin{lstlisting}
LABEL arch
        LINUX ../vmlinuz-linux
        APPEND root=UUID=978e3e81-8048-4ae1-8a06-aa727458e8ff ro
        INITRD ../initramfs-linux.img
\end{lstlisting}

Por último,
\begin{lstlisting}
# exit
\end{lstlisting}

\subsection{Reinicialização}
Antes de reinicializar, é preciso desmontar as partições:
\begin{lstlisting}
# umount /mnt/{boot,home,}
\end{lstlisting}

Para terminar, reinicialize a máquina:
\begin{lstlisting}
# reboot
\end{lstlisting}

\section{Pós-Instalação}

\section{Solução de problemas}
A seguir a solução para alguns problemas que já tivemos.

\begin{enumerate}
  \item \textbf{Máquina não consegue bootar.}

    Isso pode decorrer de alguma configuração inadequada de uma instalação
    anterior ou da BIOS. Será preciso verificar ambos os casos.

  \item \textbf{Alguma configuração na BIOS impede o boot.}

    Recomenda-se que restaure a configuração inicial da BIOS que pode ser feito
    de várias maneiras:
    \begin{itemize}
      \item Própria opção no painel de configuração da BIOS.
      \item Utilizando jumper de restauração da BIOS (será preciso abrir a
        máquina).
      \item Removendo a bateria da placa-mãe (será preciso abrir a máquina e
        talvez tenha que utilizar o jumper de restauração da BIOS).
    \end{itemize}
    Para os dois últimos ítens recomenda-se dar uma olhada no manual da
    placa-mãe da máquina.

  \item \textbf{Não consigo acesso a internet.}

    Verifique se a configuração de conexão de internet está correta. As máquinas
    de teste não utilizam DHCP.

  \item \textbf{O dispositivo de rede não é listado pelo sistema operaciona.}

    Talvez o dispositivo de rede esteja desabilitado na configuração de rede,
    verifique se este é o caso.

    O dispositivo de rede pode ter queimado.
\end{enumerate}

\section*{Referências}
A seguir encontram-se uma lista de páginas utilizadas como referência para este
documento.
\begin{itemize}
  \item Arch Linux Wiki - Beginners' Guide
    [\url{https://wiki.archlinux.org/index.php/Beginners%27_Guide}],
  \item Arch Linux Wiki - Installation Guide
    [\url{https://wiki.archlinux.org/index.php/Installation_Guide}].
\end{itemize}
\end{document}
