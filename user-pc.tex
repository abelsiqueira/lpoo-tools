\documentclass[A4paper]{article}
\usepackage[utf8]{inputenc}
\usepackage[left=2cm,right=2cm,top=3cm,bottom=3cm]{geometry}
\usepackage[brazil]{babel}
\usepackage{indentfirst}
\usepackage{url}
\usepackage{listings}

\title{LPOO - Manutenção de Máquinas de Usuário}
\author{Raniere Silva \\ \url{ra092767@ime.unicamp.br} \and
        Abel Siqueira \\ \url{abel.s.siqueira@gmail.com}}
\begin{document}
\maketitle

\section{Ambiente}
O LPOO possui disponíveis algumas máquinas para uso de seus membros.

As máquinas do LPOO utilizam como sistema operacional o GNU/Linux na forma da
distribuição Xubuntu.

Os seguintes softwares encontram-se instalados:
\begin{itemize}
  \item \lstinline+gfortran+
  \item \lstinline+gcc+
  \item \lstinline!g++!
  \item \lstinline+python3+
\end{itemize}

\section{Pré-Requisitos para Instalação}
Para configurar uma máquina será preciso:
\begin{itemize}
  \item conexão com a internet,
  \item um pen-drive vazio.
\end{itemize}

\section{Pré-Instalação}

\section{Instalação}
\subsection{Configurando teclado}
\subsection{Conectando com a Internet}
\subsection{Particionamento do disco}
\subsection{Instalando o Xubuntu}
\subsection{Reinicialização}

\section{Pós-Instalação}

\section{Solução de problemas}
Verique a solução de problemas em ``LPOO - Manutenção de Máquinas para Teste''.
\end{document}
